\leading{13pt}
% no linebreaks inside equations
\relpenalty=10000
\binoppenalty=10000
\overfullrule5pt\relax

% we have some pretty huge multiline displays, so...
\allowdisplaybreaks

%%%%%%%%%%%%%%%%%%%%%%%%%%%%%%%%%%%%%%%%%%%%%%%%%%%%%%%%%%%%
% use mathtools equation numbering utilities

% let \[ and \] be the same as \begin{equation} and \end{equation}
\makeatletter
\AtBeginDocument{%
  \let\[\@undefined
  \DeclareRobustCommand{\[}{\begin{equation}}%
  \let\]\@undefined
  \DeclareRobustCommand{\]}{\end{equation}}%
}
\makeatother
% but only print equation numbers if needed
\mathtoolsset{showonlyrefs,showmanualtags}

%%%%%%%%%%%%%%%%%%%%%%%%%%%%%%%%%%%%%%%%%%%%%%%%%%%%%%%%%%%%
% set up hyperref

\hypersetup{
  colorlinks, 
  linkcolor=-red!75!green!50, 
  citecolor=-red!75!green!50,
  urlcolor=red,
}

% set up PDF bookmarks to the first and last pages, for convenience
\AtBeginDocument{%
  \clearpage
  \bookmark[level=1, named=FirstPage]{Cover page}%
}
\BookmarkAtEnd{%
 \bookmarksetup{startatroot}%
 \bookmark[named=LastPage, level=0]{Last page}%
}

%%%%%%%%%%%%%%%%%%%%%%%%%%%%%%%%%%%%%%%%%%%%%%%%%%%%%%%%%%%%
% set up theorems and such

% we only have theorems and a question in the introduction
\newtheorem{Theorem}{Theorem}[section]
\renewcommand\theTheorem{\arabic{Theorem}}
\theoremstyle{definition}

\newtheorem{theorem}[Theorem]{Theorem}
\newtheorem{Question}[Theorem]{Question}
\newtheorem{question}[Theorem]{Question}

% and after that the rest
\newtheorem{Proposition}{Proposition}[section]

\newtheorem{proposition}[Proposition]{Proposition}

\newtheorem{lemma}[Proposition]{Lemma}
\newtheorem{Lemma}[Proposition]{Lemma}

\newtheorem{corollary}[Proposition]{Corollary}
\newtheorem{Corollary}[Proposition]{Corollary}

\newtheorem{conjecture}[Proposition]{Conjecture}
\newtheorem{Conjecture}[Proposition]{Conjecture}

\theoremstyle{definition}

\newtheorem{example}[Proposition]{Example}
\newtheorem{examples}[Proposition]{Examples}
\newtheorem{definition}[Proposition]{Definition}

\newtheorem{claim}[Proposition]{Claim}

% answer enviroment without numbering


%%%%%%%%%%%%%%%%%%%%%%%%%%%%%%%%%%%%%%%%%%%%%%%%%%%%%%

% \numberwithin{equation}{section}

% \newcommand\claim[2][.8]{%
%   \begin{minipage}[c]{#1\displaywidth}%
%   \itshape
%   #2
%   \end{minipage}%
% }

% \newmdenv[tikzsetting={draw=blue,fill=red,fill opacity=0}]{answer}


% let us make the remark style from amsthm.sty have the same pre/postskips
% as the rest of the proclamations
\makeatletter
\def\th@remark{%
  \thm@headfont{\itshape}%
  \normalfont 
  \thm@preskip\topsep % commented this out: \divide\thm@preskip\tw@
  \thm@postskip\thm@preskip
}
\makeatletter
\theoremstyle{remark}
\newtheorem{Remark}[Proposition]{Remark}
\newtheorem{remark}[Proposition]{Remark}

% use etoolbox to patch the Remark and Example environments to add a
% \lozenge at the end. We do this in this way (and not just a
% \AtEndEnvironment{theorem}{\null\hfill\qedsymbol}) so that we can use
% \qedhere in remarks...
\AtBeginEnvironment{example}{\def\qedsymbol{$\lozenge$}\pushQED{\qed}}
\AtEndEnvironment{example}{\popQED}
\AtBeginEnvironment{examples}{\def\qedsymbol{$\lozenge$}\pushQED{\qed}}
\AtEndEnvironment{examples}{\popQED}
\AtBeginEnvironment{remark}{\def\qedsymbol{$\lozenge$}\pushQED{\qed}}
\AtEndEnvironment{remark}{\popQED}

%%%%%%%%%%%%%%%%%%%%%%%%%%%%%%%%%%%%%%%%%%%%%%%%%%%%%%%%%%%%
% set up the color boxes 

% a common style for all color boxes
\tcbset{common/.style={
  breakable,
  enhanced jigsaw,
  before skip=10pt,
  colback=#1,
  grow to right by=0.25cm,
  grow to left by=0.25cm,
  if odd page={right=0.25cm,left=0.25cm}{right=0.25cm,left=0.25cm},
  toggle enlargement,
  boxsep=0pt,
  toprule=0.07mm,
  bottomrule=0.07mm,
  leftrule=0mm,
  rightrule=0mm,
  arc=0mm,
  }
}

% tcolorboxes forget the parindent, so we save its value when we start
% and reset it in each tcolorbox
\newlength{\normalparindent}
\AtBeginDocument{\setlength{\normalparindent}{\parindent}}
\tcbset{before upper={\setlength{\parindent}{\normalparindent}}} 

% colorize the theorems and such 
\colorlet{thm-color}{blue!5}
\definecolor{q-color}{HTML}{ffb703}
\colorlet{question-color}{q-color!10}
\colorlet{claim-color}{green!5}
\colorlet{remark-color}{blue!5}
\tcolorboxenvironment{claim}{common={claim-color}}       % smalls
\tcolorboxenvironment{Claim}{common={claim-color}}       % Caps

\tcolorboxenvironment{remark}{common={remark-color}}    % smalls
\tcolorboxenvironment{Remark}{common={remark-color}}    % Caps

\tcolorboxenvironment{question}{common={question-color}}   % smalls
\tcolorboxenvironment{Question}{common={question-color}}   % Caps

\foreach \t in {
  Proposition, Theorem, Lemma, Corollary, Conjecture
  proposition, theorem, lemma, corollary, conjecture}
{
  \expandafter\tcolorboxenvironment\expandafter{\t}{
      common={thm-color},
  }
}


\colorlet{answer-color}{blue!5}

% \newmdenv[%
%   skipabove=12pt,
%   skipbelow=12pt,
%   innertopmargin=12pt,
%   leftmargin=-5pt,
%   rightmargin=-5pt, 
%   innerleftmargin=12pt,
%   innerrightmargin=12pt,
%   innerbottommargin=12pt,
%   backgroundcolor={answer-color},
%   linewidth=0,
%   % everyline=true,
%   % roundcorner=3pt,
%   tikzsetting={draw=blue,fill={answer-color},fill opacity=0}
% ]{answer}%

% a common style for all color boxes
\tcbset{answer/.style={
  breakable,
  enhanced jigsaw,
  before skip=10pt,
  colback=#1,
  grow to right by=0.25cm,
  grow to left by=0.25cm,
  if odd page={right=0.25cm,left=0.25cm}{right=0.25cm,left=0.25cm},
  toggle enlargement,
  boxsep=10pt,
  toprule=.1mm,
  bottomrule=.1mm,
  leftrule=0mm,
  rightrule=0mm,
  arc=0mm,
  }
}

\newtcolorbox{answer}{answer={answer-color}}

% \tcolorboxenvironment{answer}{common={answer-color}}

% reset theorem and such counters at the beginning of each answer
\AtBeginEnvironment{problem}{%
  % \setcounter{Theorem}{0}
  % \setcounter{Proposition}{0}
  % \setcounter{Lemma}{0}
  % \setcounter{Corollary}{0}
  % \setcounter{Conjecture}{0}
  % \setcounter{question}{0}
  \setcounter{Theorem}{0}
  \setcounter{Proposition}{0}
}

% we use this for all enumerations in theorems
\newlist{thmlist}{enumerate}{1}
\setlist[thmlist]{
        nolistsep,
        ref={\mdseries\textup{(\emph{\roman*})}},
        label={\mdseries\textup{(\emph{\roman*})}},
        listparindent=\parindent
        }
\newcommand\thmitem[1]{\textup{(\emph{\romannumeral #1})}}

%%%%%%%%%%%%%%%%%%%%%%%%%%%%%%%%%%%%%%%%%%%%%%%%%%%%%%%%%%%%
% the table of contents

\makeatletter
\titlecontents{section}
  [0.8em] 
  {}
  {\textsection\thecontentslabel. }
  {}
  {\titlerule*[0.25pc]{.}\contentspage}

\renewcommand\tableofcontents{%
    \begingroup
    \centering
    {\normalsize\bfseries\contentsname}\par
    \small
    \vspace*{0.5pc}%
    \begin{minipage}{.6\textwidth}
    \@starttoc{toc}%
    \end{minipage}\par
    \endgroup
    \vspace*{2pc}
    }
\makeatother

%%%%%%%%%%%%%%%%%%%%%%%%%%%%%%%%%%%%%%%%%%%%%%%%%%%%%%%%%%%%
% captions and floats

\captionsetup{skip=1pc}
\setlength{\textfloatsep}{3pc plus 5.0pt minus 5.0pt}

% normal itemizations, with spaced items consisting of paragraphs,
\setlist[itemize]{itemsep=1ex, listparindent=1em, parsep=0pt, label=\textbullet}

%%%%%%%%%%%%%%%%%%%%%%%%%%%%%%%%%%%%%%%%%%%%%%%%%%%%%%%%%%%%
% Header BOX
\newcommand{\handout}[6]{
  \noindent
  \begin{center}
  \setlength{\fboxrule}{1.2pt}
  \framebox{
    \vbox{
      \vspace{2mm}
      \hbox to 5.78in { \textbf{#6} \hfill {\bf #2} }
      \vspace{4mm}
      \hbox to 5.78in { {\Large \hfill {\textbf{ #5 }}  \hfill} }
      \vspace{2mm}
      \hbox to 5.78in { {\textit{\textbf{#3 \hfill #4}}} }
      \vspace{1.5mm}
    }
  }
  \setlength{\fboxrule}{0.2pt}
  \end{center}
  \vspace*{4mm}
}

% Header BOX
\newcommand{\PSET}[5]{\handout{#1}{#2}{Prof.\ #3}{Student: #4}{PSET #1}{#5}}

\newcommand{\homework}[5]{\handout{#1}{#2}{Prof.\ #3}{Student: #4}{Homework assigned #1}{#5}}

\newcommand{\exam}[5]{\handout{#1}{#2}{Prof.\ #3}{Student: #4}{Exam #1}{#5}}

\newcommand{\midterm}[5]{\handout{#1}{#2}{Prof.\ #3}{Student: #4}{Mid-Term Exam #1}{#5}}

\newcommand{\reading}[5]{\handout{#1}{#2}{Prof.\ #3}{Student: #4}{Reading assigned #1}{#5}}

\newcommand{\quiz}[5]{\handout{#1}{#2}{Prof.\ #3}{Student: #4}{Quiz #1}{#5}}

\newcommand{\statement}[5]{\handout{#1}{#2}{Prof.\ #3}{Student: #4}{Participation Statement #1}{#5}}

\newcommand{\final}[5]{\handout{#1}{#2}{Prof.\ #3}{Student: #4}{FINAL EXAM --- #1}{#5}}
% Credit Statement
\newcommand{\CreditStatement}[1]{
  \noindent
  \begin{center} {
    \bf Credit Statement
  }
  \end{center}
  { #1 }
}

%%%%%%%%%%%%%%%%%%%%%%%%%%%%%%%%%%%%%%%%%%%%%%%%%%%%%%%%%%%%
% Problem Counter
\newcounter{problem}
\setcounter{problem}{0}

\newenvironment{problem}[1][]%
{%
\ifx&#1&%
  % #1 is empty
  \stepcounter{problem}
\else
  % #1 is nonempty
  \setcounter{problem}{#1}
\fi
\setcounter{section}{\value{problem}}
\setcounter{equation}{0}
\vspace{.2cm} \noindent {\center  \textbf{\\ Problem \arabic{problem}.\\}} {}~%
}{%
% \vspace{.2cm}%
}


\newcommand\isom{\mathrel{\stackon[-0.1ex]{\makebox*{\scalebox{1.08}{\AC}}{=\hfill\llap{=}}}{{\AC}}}}
\newcommand\nvisom{\rotatebox[origin=cc] {-90}{$ \isom $}}
\newcommand\visom{\rotatebox[origin=cc] {90} {$ \isom $}}


% matrices -- vertical separators
\makeatletter
\renewcommand*\env@matrix[1][*\c@MaxMatrixCols{ c}]{%
  \hskip -\arraycolsep{}
  \let\@ifnextchar\new@ifnextchar{}
  \array{#1}}
\makeatother

%%
\lstset{frame=tb,
  backgroundcolor=\color{backcolour},   
  commentstyle=\color{codepurple},
  keywordstyle=\color{NavyBlue},
  numberstyle=\tiny\color{codegray},
  stringstyle=\color{codepurple},
  basicstyle=\ttfamily\footnotesize\bfseries,
  breakatwhitespace=false,         
  breaklines=true,                 
  captionpos=t,                    
  keepspaces=true,                 
  numbers=left,                    
  numbersep=5pt,                  
  showspaces=false,                
  showstringspaces=false,
  showtabs=false,                  
  tabsize=2,
  % escapeinside={\%*}{*)},          % if you want to add LaTeX within your code
}

\tikzset{
  between/.style args={#1 and #2}{
    at = ($(#1)!0.5!(#2)$)
  }
}
\usetikzlibrary{
  automata,
  positioning,
  arrows,
  cd,
  shapes.geometric,
  calc
}

\tikzset{
  between base/.style args={#1 and #2}{
    between=#1.base and #2.base
  }
}

\colorlet{node-color}{gray!10}

\tikzset{
  ->, % makes the edges directed
  >=stealth, % makes the arrow heads bold
  node distance=3cm, % specifies the minimum distance between two nodes. Change if necessary.
  every state/.style={thick, fill={node-color}}, % sets the properties for each ’state’ node
  initial text=$ $, % sets the text that appears on the start arrow
}

\tikzstyle{startstop} = [circle, minimum size = 1cm, text centered, draw = black, fill={node-color}]
\tikzstyle{addition} = [rectangle, minimum size = 1cm, text centered, draw = black, fill={node-color}]
\tikzstyle{subtraction} = [diamond, minimum size = 1cm, text centered, draw = black, fill=node-color]
\tikzstyle{blackbox} = [rectangle, minimum size = 1cm, rounded corners, text centered, draw = black, fill={node-color}]
% \tikzstyle{arrow} = [thick,->,>=stealth]

% for package parskip
\tcbset{
  parskip/.style={
    before={\par\pagebreak[0]\parindent=0pt},
    after={\parfillskip=0pt\par},
  },
}


% algorithm loop
\SetKwBlock{Loop}{loop}{EndLoop}
